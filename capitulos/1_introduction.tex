\chapter{Introduction}
\label{introduction}

In this document we will address the enhancement of mixing in active fluids from an experimental point of view. 

Active fluids will be explained in further detail later in this document, but, for the time being, we will describe them as suspensions whose constituent elements can self-propel. Such elements can be either biological or synthetic. The former, on which we will focus in the following, consist of micro-organism populations.

These suspensions have raised the interest of the scientific community for decades, as their dynamics accounts for:

\begin{itemize}
	\item alteration of the \textbf{rheological properties},
	\item \textbf{enhanced mixing} of the surrounding fluid and, therefore, of the chemicals within it, and
	\item \textbf{increased nutrient uptake in the case microorganisms}.
\end{itemize}

In order to be able to understand and, eventually, exploit such systems, different analytical and numerical models have been developed over the last century. These modeling efforts, whose outlines will be presented in this document, need experimental confirmation.

\textbf{In the present undertaking we study the influence of the unicellular green alga \textit{Chlamydomonas reinhardtii} on the diffusion of micrometric tracers in a 2-dimensional environment.} 
Every stage of the procedure, from the experiment design and device construction, to the image processing and data analysis, is to be described in this document.

This text is organized as follows: first, we further detail the objectives of this research; second,  in Section 3, we explain some fundamental concepts for our research in the contexts of fluid mechanics, active suspensions and Brownian motion; then we introduce the state-of-the-art of both, numerical and experimental approaches to our problem.  To end with, Sections Materials and Methods, Results and Analysis, and Conclusions (respectively Sections 5, 6 and 7), summarize the experimental work carried out. 