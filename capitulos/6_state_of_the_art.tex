\chapter{State of the art}
\label{SOA}

This section constitutes an extremely brief summary of the latests studies (to our knowledge). For further detail, we recommend the reading of annual reviews on the topic such as the one of Saintillan~\cite{Saintillan}.

\section{Enhancement of diffusion and mixing: pushers and pullers}

The flow fields induced by microorganisms, improve transport and, therefore, mixing in suspensions. The way in which they do that depends, among other things, on their swimming patterns. As we have already stated, isolated swimmers generate different flow fields depending on if they are pullers or pushers. 

Orientation decorrelation mechanisms are as important for mixing as the way of swimming. These mechanisms are mainly of two types: \textit{run-and-tumble} motion and \textit{rotational diffusion}. The former, usually depending on the microswimmers themselves, consists of straight runs that alternate with random tumbles with a certain mean frequency. The latter is a combination of thermal Brownian motion, whose contribution is weak, and other sources of noise such as flagellar beating patterns.

When in groups, the perturbations exerted of the swimmers on the fluid overlap. This can have different effects depending on the level of concentration. Moreover, hydrodynamic interactions that can take place can even lead to coherent motion\footnote{This meso-scale structures have been studied for concentrated bacterial cells, but not for algal cells}, substantially changing the flow field. Interactions between microorganisms have been studied by several authors, among which we would like to cite Ishikawa (~\cite{Ishikawa},~\cite{Ishikawa9}), who, with a bottom-up approach, studied a wide range of possibilities, from interactions between two individuals to a continuum suspension level.

In spite of the apparent differences, several studies on pushers and pullers show that the dynamics of spherical tracers are qualitatively similar in both cases (see, for example, Wu and Libchaber~\cite{Wu} and Leptos et al.~\cite{Leptos}). In 2016, however, Yang et al.~\cite{Yang} would show that ellipsoidal tracers behave differently (in their rotational degree of freedom), depending on the nature of the present active particles, allowing their distinction.

Another important factor is dimensionality: in a 2D confinement such as ours, the pysical model for pushers and pullers explained in section ¿¿¿3.1.2??? remains valid. However, the dominant solution, a Stresslet, presents a larger field of influence. In 3D the velocity field generated by a Stresslet is $\propto r^{-2}$, whereas in 2D, it is $\propto r^{-1}$. Our reference article on this topic is the one of Kurtuldu et al.~\cite{Kurtuldu2011}, where the 2D environment is limited by free surfaces in order to avoid no-slip conditions. 

\section{Modelling microswimmers: an outline}

An introduction to the analytical approach has already been presented in section ¿¿¿3.1.2???. This section aims to give an insight to the numerical methods, that, based on the analytical ones, allow the analysis of greater systems, different levels of concentration...

Although continuum models are, of course, possible and have been developed (see Toner et al.~\cite{Toner}), the natural choice for the numerical description of active suspensions are particle-based simulations. 

Continuum models generally rely on the description of far-field hydrodynamic interactions, which limits their validity to relatively dilute suspensions.

In particle-based simulations, microswimmers are immersed in a continuous medium and the dynamics of each one of them is resolved in a highly coupled system. The description of individual swimmers includes high-order singularities due to particle size (i.e. close-field hydrodynamic interactions), which remain exceptional in the continuum models. The resulting positions and orientations finally allow us to depict the whole dynamics of the suspension.

The choice of the model for individual swimmers can greatly affect the output of the system simulations. Some candidates for modelling the individual behavior are the spherical squirmer model of Lighthil and Blake~\cite{Blake}, slender-body theory~\cite{batchelor_1970} of rod-like swimmers and point force distributions, such as the three-Stokeslets model proposed by Drescher et al.~\cite{Drescher2010}.

Delmotte et al.~\cite{Delmotte2015} have developed a numerical method based on the squirmer model, that has demonstrated great versatility and efficiency, taking into account short range effects such as steric interactions. 

This method \textit{extends the force-coupling method (FCM) to active particles by introducing the regularized singularities in the FCM multipole expansion that have a direct correspondence to the surface velocity modes of the squirmer model}.

In the FCM, the effect of particles on the fluid phase is represented by a localized body force that transmits to the fluid, thus representing an extra term in the Navier-Stokes equations. This force is described by a regularized multipole expansion, whose terms (monopoles, dipoles...) decay with the distance to its source, but do not tend to infinite at the source point (Gaussian shapes are usually chosen). This allows the total particle force, the one associated with constraints included, to be projected onto a structured and simple grid over which an efficient parallel Stokes solver can simultaneously process all the hydrodynamic interactions and eventually find the whole fluid field.

\textbf{Recently, Delmotte et al.~\cite{Delmotte2018} have used this method to perform simulations whose goal is to examine tracer displacements and effective tracer diffusivity in squirmer suspensions. Our experiments are partially aimed at validating the latest upgrade of this model, which consists in simulating free-surface type boundary conditions.}


