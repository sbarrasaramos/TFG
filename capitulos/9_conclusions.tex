\chapter{Conclusions}
\label{conclusions}

In this section we will go through the posed objectives and asses the level of observance for each of them.

\section{Acquaintance with bibliography on the matter}

The cited papers are very few compared to the many we have dealt with, and much fewer compared to the existing bibliography on the topic. Consequently, full compliance on this matter is impossible to achieve. Yet, we have found a lot of the answers to our questions among the pages of the cited documents, which has saved us a lot of time.

\section{Algae culture and handling}

In spite of some contamination related issues, we have managed to maintain a constant supply of \textit{Chlamydomonas reinhardtii}, allowing the realization of experiments on a regular basis. 

\section{Device design and construction}

The initial development of our was a quick process: in less than two weeks we were already performing the first experiences. However, upgrades to this initial idea have been implemented all throughout the project.

Unfortunately, at a certain point, the device was damaged and a temporary solution had to be devised. This improvised solution turned out to be rather effective and even inspired some of the protocols that we would finally apply to deal with the problem of \textit{background flows}.

\section{Image acquisition}

Although we are indeed using fluorescent beads, the intensity of the autofluorescence of \textit{Chlamydomonas reinhardtii}, provided by its chlorophyll, is not enough to consider fluorescence microscopy as an option.

By means of light microscopy, in the end we seem to have managed to obtain at least two sets of images where the motion of algae and tracers is not-biased, either by background flows or other factors. Yet, probably because of the noise, we have not been able to properly analyze very small displacements with the selected tracking algorithm. This means that further image processing is to be done and tested.

The selected technique is particle tracking velocimetry (PTV), but the resulting images should also be valid for other velocity measuring techniques such as particle image velocimetry (PIV). This technique (PIV) should be explored in future works to find out its advantages and disadvantages, compared to PTV, when applied to the analysis of our images.

\section{Particle tracking}


\textcolor{red}{We intend to develop a Python code that will allow an automation of the image processing needed to improve the performance of the selected tracking algorithm. Such algorithm, originally implemented by John Crocker and Eric Weeks~\cite{Crocker} in Interactive Data Language (IDL) has been reimplemented in a Python
open library under the name Trackpy. Our code will call on that library and save the trajectories and other derived data in text files to be analyzed later.}

\section{Data analysis}

The information revealed up to now by our data seems to be in agreement with previous literature on the topic (mainly~\cite{Kurtuldu2011}). Nonetheless, further analysis of the data is still required. 

Once this analysis will be completed, a similar one must be conducted on data obtained from thicker layers, since the numerical model is currently unable to deal with 2D systems\footnote{For systems under a certain, the fictitious forces playing the role of boundary conditions cane explode numerically when two swimmers meet.}.

\section{Side phenomena explanation}

Three side phenomena have been of our main interest, since they prevented us from acquiring exploitable images: background flows, quiescent beads and premature algal death/loss of flagella.

The first of this phenomena has been explained in this document and successfully mitigated. The second one, although explicable, seems to be unavoidable in our experiments, so we need to make sure it does not affect our results. The third one has not yet been understood, but can be avoided in a rather simple way.

\section{Conclusions and open-ended questions}

\textcolor{red}{¿¿¿ Pushers ???}

\textcolor{red}{This project is headed to draw a set of qualitative and quantitative conclusions concerning the enhanced mixing generated by \textit{Chlamydomonas reinhardtii} in a 2D environment. It is also our intention to transfer the acquired knowledge to 3D configurations in order to find out the differences induced by dimensionality. Last, but not least, the emerging and unanswered questions will be put on paper to inspire future works.}