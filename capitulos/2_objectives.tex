\chapter{Objectives}
\label{objectives}

In an experimental research undertaking and with a very limited period of time ahead it is always hard to define an attainable list of objectives from the beginning. Hence its constant evolution. Nonetheless, the objectives can be summarized as follows:

\section{Acquaintance with bibliography on the matter}

As it has already been mentioned, this project merges several disciplines such as fluid mechanics, biology or chemistry. Therefore, the number and diversity of papers and books employed along this project is rather high despite the relatively short duration of the project.

To mention one of them, the paper of Kurtuldu~\cite{Kurtuldu2011} served as an introduction to the topic and has also been used as a reference when it comes to asses the obtained results.

The rest of the papers will be cited along this report in the pertinent sections.

\section{Algae culture and handling}

To ensure their survival, size, etc., \textit{Chlamydomonas reinhardtii} needs to be cultivated under certain conditions.

Different conditions, that will be further specified in this document, must be considered for several reasons.

Other handling aspects, regarding for example, concentration or sterile manipulation, will also be illustrated. 

\section{Device design and construction}

Establishing and maintaining a thin film of the studied solution requires of a specific device. 

This thin film must also be observable with a microscope. That imposes, for instance, constraints on the size of the device, as well as on the film durability.

A number of other requirements and problems have emerged during the development of the project that have pushed us to devise several versions of the mechanism.

\section{Image acquisition}

In order to obtain representative statistics, we need to have a sufficient record of images, where the motion of algae and tracers is not-biased and clear enough to be tracked.

Consequently, we will face a choice between light microscopy and fluorescence microscopy.

The selected technique is particle tracking velocimetry (PTV), but the resulting images should also be valid for other velocity measuring techniques such as particle image velocimetry (PIV).

\section{Particle tracking}

We intend to develop a Python code that will allow an automation of the image processing needed to improve the performance of the selected tracking algorithm.

Such algorithm, originally implemented by John Crocker and Eric Weeks~\cite{Crocker} in Interactive Data Language (IDL) has been reimplemented in a Python open library under the name Trackpy.

Our code will call on that library and save the trajectories and other derived data in text files to be analyzed later. 

\section{Data analysis}

In the last place, the resulting data must be analyzed in a quantitative (statistical) way. The outcome is meant to help validate or, if needed, re-orient the numerical models that are being developed on the topic within the framework of the LadHyX.

To this end, we will develop another Python application.

All the code is to be documented and will be available for future studies on this or similar topics.

\section{Side phenomena explanation}

In an almost parallel way with the bibliography, we will try to explain phenomena directly and indirectly related to the subject of study.

Most of this phenomena have emerged as problems to solve in order to obtain clear, analyzable data. This requires, at least, of a qualitative description, and, sometimes, if possible, of a quantitative one.

\section{Conclusions and open-ended questions}

This project is headed to draw a set of qualitative and quantitative conclusions concerning the enhanced mixing generated by \textit{Chlamydomonas reinhardtii} in a 2D environment.

It is also our intention to transfer the acquired knowledge to 3D configurations in order to find out the differences induced by dimensionality.

Last, but not least, the emerging and unanswered questions will be put on paper to inspire future works.
















