\chapter*{Abstract}

In this document we address the topic of enhanced diffusion and mixing in biological active suspensions from an experimental point of view. To work with, we have chosen a very specific configuration: 2D films containing low concentrations of the unicellular green alga \textit{Chlamydomonas reinhardtii}.

To this end, a specific device that can hold a thin water film has been designed and built, on the basis of previous works on similar topics (see ~\cite{Sokolov2007}, ~\cite{Guasto}). \textit{Chlamydomonas reinhardtii} has been cultivated by ourselves in the laboratory as well.

The film, containing the microswimmers as well as micron-sized spherical tracers, has been studied with the help of an optical microscope, under a certain set of conditions to avoid biases in the measures. The videos resulting from a high-frequency imaging have been processed and analyzed by means of the particle-tracking method. 

This technique has been applied to both algae and tracers, providing statistically significant data samples for each video. Several algorithms have been developed that enable the semi-automatic image processing and analysis.

The resulting data have been exploited in several ways that allow the comparison with previous papers (see ~\cite{Kurtuldu2011}). 

The ultimate purpose of this experimental project is to validate a numerical model currently being developed by the team of Dr. Delmotte. This model extends previous works (see ~\cite{Delmotte2015}, ~\cite{Delmotte2018}) by implementing free surfaces as boundary conditions.

Either by reducing the computational domain or by thickening the films used through our experiments, we will be able to assess this model's accuracy.
