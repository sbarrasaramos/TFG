\chapter*{Preamble}
\thispagestyle{empty}
Our efforts during this months have been driven toward the description and quantification of the biomixing generated by microswimmers, more specifically by the unicellular alga \textit{Chlamydomonas reinhardtii}, in 2D.

Several are the reasons that have prompted this investigation and, more precisely, the experimental work that is to be described in the following document.

With the evolution of classical sciences probably arriving to an asymptote, the interfaces between different fields have aroused the interest of many. In our specific case, the transmission of analytical techniques from fluid mechanics, the reinterpretation of data in a statistical way, and a shift in perspective can help us understand or, at least, justify the behavior of individuals and communities of \textit{C. reinhardtii}. This could, for instance, help biologists devise ecosystem models which could, in turn, end up influencing other disciplines and so on.

Regarding more down-to-earth objectives, and in spite of the basic character of this research, we hope that a better understanding of active suspensions will help, among other things, improve the performances of some state-of-the-art devices, such as algae bioreactors or \textit{Lab-on-a-chip} devices for drug delivery.

At a personal level, one of the main arguments for choosing this project has been the opportunity to work hand in hand with the team developing the corresponding simulations. The limited amount of time available would have prevented a single intern from tackling the different aspects of the project, leaving an important gap in the intern's formation and the project itself.

The latter has turned out to be more than a good reason, as the fruitful discussions with the members of the LadHyX and the Institut Pasteur have helped us solve the endless issues that are unfailingly associated to experimental research.

\cleardoublepage %salta a nueva página impar

\chapter*{Acknowledgements}

\thispagestyle{empty}
\vspace{1cm}

The completion of this project would not have been possible without the assistance and stimulus of so many people, that I should probably speak about \textit{teams}.

First of all, the team in charge of this project, within which I must mention my tutor, Dr. Gabriel Amselem, as well as Dr. Blaise Delmotte and Axel Ivaldi. It has been a pleasure and an honor to go with them through my first research experience.

Secondly, I want to thank what I am going to call the \textit{Building 65} team, which is in fact composed of several teams, and whose members have introduced me to biology and biomechanics, deepening my incipient interest in those fields. 

I cannot forget thanking the people in Institut Pasteur, in particular, Dr. Charles Baroud, whose ideas have helped us solve or circumvent the problems we were facing.

Last but by no means least, I want to thanks the Spanish Governments of the last decades (maybe centuries) for their modest investments in science and culture. This has encouraged me and many others to explore the world, providing us with a more than solid argument for convincing our families and friends to expect us back later than sooner.

\cleardoublepage %salta a nueva página impar

% Aquí va la dedicatoria si la hubiese. Si no, comentar la(s) linea(s) siguientes
\chapter*{}
\setlength{\leftmargin}{0.5\textwidth}
\setlength{\parsep}{0cm}
\addtolength{\topsep}{0.5cm}

\begin{flushright}
	\small\em{
		To my family and true friends,\\
		who have remained by my side even in the distance}
\end{flushright}


\cleardoublepage %salta a nueva página impar

% Aquí va la cita célebre si la hubiese. Si no, comentar la(s) linea(s) siguientes
\chapter*{}

\setlength{\leftmargin}{0.5\textwidth}
\setlength{\parsep}{0cm}
\addtolength{\topsep}{0.5cm}
\begin{flushright}
	\small\em{
		I don't know where I'm going from here,\\ 
		but I promise it won't be boring
	}
\end{flushright}

\begin{flushright}
	\small{
		David Bowie.
	}
\end{flushright}

\cleardoublepage %salta a nueva página impar

